\documentclass[9pt,twocolumn,twoside,]{pnas-new}

% Use the lineno option to display guide line numbers if required.
% Note that the use of elements such as single-column equations
% may affect the guide line number alignment.


\usepackage[T1]{fontenc}
\usepackage[utf8]{inputenc}

% tightlist command for lists without linebreak
\providecommand{\tightlist}{%
  \setlength{\itemsep}{0pt}\setlength{\parskip}{0pt}}




\templatetype{pnasresearcharticle}  % Choose template

\title{Dynamique des communautés}

\author[a]{Kyara Boisvert}
\author[a]{Zoé Chol}
\author[a]{William Simard}

  \affil[a]{Université de Sherbrooke, Département d'écologie, 2500
Boulevard de l'Université, Sherbrooke, Québec, J1N 3C6}


% Please give the surname of the lead author for the running footer
\leadauthor{}

% Please add here a significance statement to explain the relevance of your work
\significancestatement{}


\authorcontributions{}



\correspondingauthor{\textsuperscript{} }

% Keywords are not mandatory, but authors are strongly encouraged to provide them. If provided, please include two to five keywords, separated by the pipe symbol, e.g:
 \keywords{  abondance |  taxonomie |  optional |  optional |  optional  } 

\begin{abstract}
Please provide an abstract of no more than 250 words in a single
paragraph. Abstracts should explain to the general reader the major
contributions of the article. References in the abstract must be cited
in full within the abstract itself and cited in the text.
\end{abstract}

\dates{This manuscript was compiled on \today}
\doi{\url{www.pnas.org/cgi/doi/10.1073/pnas.XXXXXXXXXX}}

\begin{document}

% Optional adjustment to line up main text (after abstract) of first page with line numbers, when using both lineno and twocolumn options.
% You should only change this length when you've finalised the article contents.
\verticaladjustment{-2pt}



\maketitle
\thispagestyle{firststyle}
\ifthenelse{\boolean{shortarticle}}{\ifthenelse{\boolean{singlecolumn}}{\abscontentformatted}{\abscontent}}{}

% If your first paragraph (i.e. with the \dropcap) contains a list environment (quote, quotation, theorem, definition, enumerate, itemize...), the line after the list may have some extra indentation. If this is the case, add \parshape=0 to the end of the list environment.

\acknow{}

À travers le temps, nos écosystèmes se sont formés grâce à une grande
diversité d'espèces animales et végétales. Cette richesse spécifique,
retrouvé sur Terre, est intrinsèquement et de manière pragmatique
essentiel à la survie des êtres humains (Kissling, 2018). Malgré la
reconnaissance de ces faits et des efforts de conservations constants au
maintient de la biodiversité, elle continue de décliner à un rythme
alarmant (Kissling, 2018; Newbold, 2016). Plusieurs populations
largement répandues ainsi que des espèces menacées sont en déclins
(Tittensor, 2014). Combiné avec l'exploitation humaine des écosystèmes
terrestres et des écosystèmes marins, ses facteurs en résultent en un
biotope Terrestre qui est utilisé au delà de sa durabilité (Kissling,
2018). À la suite d'analyses statistiques et d'observations d'une banque
de données d'une série temporelle comportant des échantillonnages entre
les années 1950 et 2020, incluant une variété de taxon végétal et
animal, nous en sommes arrivé à trois questions de recherche. La
première est : Observe-t-on un déclin de la biodiversité à travers les
années? La deuxième question est : Quel taxon a-t-on le plus (ou le
moins) observé à travers les années? Puis la dernière est : Quel taxon a
le plus décliné depuis les années 1970? Pour répondre à ces questions,
nous avons effectuées d'autres analyses statistiques. Elles consistaient
l'abondance des différentes espèces retrouvées à travers les années.
Avec la répartition de l'abondance des espèces, elle nous permet
d'obtenir des preuves sur le niveau de rareté (ou non) d'une espèce
particulière selon d'autres espèces (Matthews, 2015).''

\section{Méthode}\label{muxe9thode}

Afin de réaliser les analyses statistiques, la banque de données, nommée
séries temporelles, a été fourni par Biodiversité Québec. Ces données
ont été récoltées sur plusieurs décennies, de 1950 à 2020. Cette banque
nous a permis de réaliser plusieurs analyses statistiques. Pour débuter,
nous avons importé les données taxonomiques dans le logiciel R. Par la
suite, nous avons procédé par un nettoyage de l'information de manière à
avoir une base précise. Ensuite, une analyse statistique a été fait dans
le but de créer des data frame contenant les éléments, tel que les noms
scientifiques, les années, les coordonnées, les sources qui procuraient
l'information, etc. Ces data frame ont permis de créer de nouvelles
bases de données utilisées dans logiciel SQL de R studio. Puis ces
nouveaux data frame ont été utilisés pour créer les figures trois
figures suivantes. Le premier graphique a été conçu à l'aide de la
fonction ``plot''. Le type de lignes, l'épaisseur de la ligne ainsi que
la couleur de celle-ci ont été défini selon les caractéristiques
désirées. Ce graphique représente la richesse spécifique par année selon
le nombre d'espèces observées. La figure deux, est un graphique à bar
(fonction ``bar plot'') qui représente les différentes classes
taxonomiques ainsi que le nombre d'observation sur une échelle
logarithmique. Des caractéristiques spécifiques tel que la couleur des
bandes et leurs bordures ont été attribués. Et la troisième figure, est
aussi un graphique à bar représentant les variations, en pourcentage,
des espèces observées selon les différentes classes à partir des années
1970. Les bandes rouges représentent les diminutions en abondances des
espèces et les bandes vertes sont les augmentations depuis 1970 selon
chaque classes taxonomiques observées.''

\section{Résultats}\label{ruxe9sultats}

Les analyses ont démontré, pour la figure 1, qu'il y a eu une
augmentation dans la richesse des espèces entre les années 1950 et
\textasciitilde{} 1993. De plus, nous observons, sur la figure 1, une
forte augmentation du nombre d'espèces observées lors des années 1970.
Cependant, nous pouvons aussi voir un déclin dans l'abondance des
espèces à partir de l'année 1995. À la suite de cette année, 1995, il y
a quelques hausses d'espèces observées, mais de façon générale on
remarque un déclin qui s'aggrave d'années en années. Ce qu'on constate
pour la figure 2, sont les observations par classes taxonomiques sur une
échelle logarithmique. La classe qui a le plus grand nombre total
d'observations est les Teleosteis. En revanche, celle qui en a le moins
est les Petromyzontis. Pour la troisième figure, on observe les
variations, en pourcentage, des différentes classes taxonomiques depuis
les années 1970. Les bandes vertes sont des valeurs positives démontrant
une augmentation d'espèces de cette classe et les bandes rouges sont des
diminutions dans le nombre d'espèces observées, donc des valeurs
négatives. Pour cette figure, la classe des Amphibia est celle qui a la
plus grande variation (500\%). Les classes Chondrostei et Equisetopsida
ont une valeur respective de 100\%. Dans les valeurs négatives, on
constate que la classe Aves est celle qui a la plus grande diminution
avec une valeur de - 99\%. Nous pouvons aussi observer une valeur de 0
pour la classe des Petromyzontis.''

\section{Discussion}\label{discussion}

\section{Conclusion}\label{conclusion}

\section{Bibliographie}\label{bibliographie}

This PNAS journal template is provided to help you write your work in
the correct journal format. Instructions for use are provided below.



% Bibliography
% \bibliography{pnas-sample}

\end{document}
